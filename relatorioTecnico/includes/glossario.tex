\begin{table}[h]
\caption{\small{Glossário para o projeto Wall-e.}}
\begin{tabular}{p{2.5cm}|p{12cm}}
\hline
TERMO & SIGNIFICADO \\
\hline
ROS                         & \textit{Robot Operating System} - Um \textit{framework} que contém um conjunto de bibliotecas e ferramentas para facilitar o desenvolvimento de trabalhos científicos e industriais em robôs. Este \textit{framework} utiliza um modelo de tópicos, com publicadores e assinantes. Os assinantes monitoram mudanças dentro dos tópicos e os publicadores podem “publicar” valores dentro dos tópicos. \\
TÓPICOS                     & São uma representação no \textit{software} de “barramentos nomeados” pelos quais nós (nodos) trocam mensagens. Cada tópico pode ter vários assinantes e vários publicadores, ou seja, vários programas podem ler e vários podem escrever em cada tópico.                                                                                                                            \\
NÓS                         & Os nós são programas escritos em C++ ou Python que necessitam de algumas funcionalidades do ROS para funcionar e se comunicam uns com os outros por meio de tópicos.                                                                                                                                                                                                       \\
ROSCORE                     & É uma coleção de nós e programas que são pré-requisito de um sistema baseado no ROS. Como o próprio nome diz, é o núcleo do ROS.                                                                                                                                                                                                                                           \\
SLAM                        & \textit{Simultaneous Localization and Mapping} - localização e mapeamento simultâneos. Diz respeito a uma técnica que será usada nesse projeto para localizar e mapear o ambiente simultaneamente.                                                                                                                                                                                  \\
ODOMETRIA                   & Método utilizado para estimar a posição do robô. Tal método se utiliza de informações incrementais ao longo do tempo e acaba possuindo uma grande acumulação de erros, sendo necessário algum método adicional para minimizar este fato. No caso deste projeto é dado por um encoder nas rodas do robô.                                                                    \\
GMAPPING                    & Sistema de localização que constrói um mapa bidimensional (semelhante a uma planta baixa) que é utilizado como parte integrante do SLAM.                                                                                                                                                                                                                                   \\
OCTOMAP                     & Uma biblioteca para mapeamento 3D de ambientes feito especificamente para robótica. Com ela é possível criar uma visualização 3D de um ambiente a partir de dados de profundidade do Kinect. Também usado como \verb|octomap| para fazer referência ao pacote.                                                                                                                                                                               \\
NAVIGATION STACK            & Conjunto de pacotes do ROS que provê diversas funcionalidades necessárias para realizar SLAM.                                                                                                                                                                                                                                                                              \\
WIIMOTE                     & Controle do console Wii, que possui um acelerômetro e giroscópio. Utiliza-se de comunicação \textit{bluetooth} para transferência de seus dados. Seu \textit{feedback} pode ser realizado por LEDs na frente do controle e por vibração do mesmo.Também usado como \verb|wiimote| para fazer refêrencia ao pacote.                                                                           
\\
KINECT                      & Sensor com câmera e infravermelho, sendo possível extrair imagens RGB, P\&B e uma imagem semelhante a um mapa de calor, onde o gradiente representa a distância.                                                                                                                                                                                                           \\
FRAMEWORK                   & Neste caso ele é uma camada essencial onde os programas são construídos (ver: http://whatis.techtarget.com/definition/framework)                                                                                                                                                                                                                                           \\
POINTCLOUD                  & Nuvem de pontos definidos em três dimensões, ou seja, cada ponto é representado por X, Y, Z. Ele é gerado pelo Kinect.                                                                                                                                                                                                                                                     \\
MOVE BASE                  & Esse pacote é o responsável por criar os mapas de custo a partir das fontes de observação e também por traçar as trajetórias, a partir de \textit{plugins}.
\\
DWA LOCAL PLANNER         & Um dos \textit{plugins} que traça trajetórias locais para a navegação autônoma.
\\
PACOTE DO ROS               & Ele é um diretório e também a forma mais atômica de se utilizar um \textit{software}.
\end{tabular}
\label{tab:glossario}
\end{table}