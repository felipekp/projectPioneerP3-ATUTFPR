A equipe acredita que todos os objetivos da disciplina foram cumpridos, desde a comunicação sem fio, como o trabalho com \textit{hardware} e \textit{software}. Tivemos resultados favoráveis e todas as melhorias realizadas e documentadas irão auxiliar ao longo do desenvolvimento do projeto nos semestres que seguem, considerando que este trabalho foi uma extensão da iniciação científica de um dos integrantes, sentimos que o aspecto de proporcionar vias de melhorias futuras é muito interessante, além da documentação presente aqui para possível compartilhamento com a comunidade.

Um dos focos do grupo foi a abordagem de como as limitações dos sensores utilizados, principalmente do Kinect, podem ser superadas de maneiras menos custosas, buscando sempre uma relação custo-benefício alta. Podemos perceber isso analisando o custo dos equipamentos presentes no ínicio do projeto e o custo de nossas melhorias, disponibilizados na Tabela \ref{tab:custoAnterior} e o custo envolvido no aprimoramento deste sistema está na Tabela \ref{tab:custoNosso}.


\begin{table}[h]
\caption{\small{Custo dos equipamentos já presentes no projeto.}}
\begin{center}
\begin{tabular}{c|c}
\hline
Descrição & Valor em reais \\
\hline
Kinect & 300,00 \\ 
Controle de Wii e adaptador \textit{bluetooth} & 100,00 \\  
Robô Pioneer 3-AT & 18.000,00 \\ 
\textit{Netbook} & 700,00 \\
Arduino Mega 2560 & 90,00 \\
\hline
Total & 19.190,00 \\
\hline
\end{tabular}
\end{center}
\label{tab:custoAnterior}
\end{table}

\begin{table}[h]
\caption{\small{Custo para o aprimoramento.}}
\begin{center}
\begin{tabular}{c|c}
\hline
Descrição & Valor em reais \\
\hline
Materiais para a nova placa do Arduino & 10,00 \\ 
Servo motor MG996R & 55,00 \\  
Estrutura para o Kinect & 15,00 \\ 
\textit{Hub} USB 2.0 & 10,00 \\
\hline
Total & 80,00 \\
\hline
\end{tabular}
\end{center}
\label{tab:custoNosso}
\end{table}

Ao compararmos o custo total de equipamentos que já estavam presentes no projeto ($19.190,00$ reais) e o custo total para o aprimoramento ($80,00$ reais) percebemos como o pequeno investimento monetário tornou possível adicionar as funções de: teleoperação com um Wiimote do robô e da posição da câmera do Kinect, mapeamento 3D, assim como melhorias na navegação autônoma. 

Desta forma, acreditamos que o trabalho realizado apresenta muitas opções para o futuro do projeto como um todo, algo muito vantajoso quando se pensa na natureza de melhoria contínua do mesmo. Acreditamos que com a descoberta de pacotes como o \verb|ira_laser_tools| e a utilização de um computador com melhores configurações é possível gerar resultados ainda melhores.

Outrossim, por mais que ainda existam inconvenientes na navegação autônoma, como falta de repetibilidade, ou até mesmo erros de má localização, houve melhoras signficativas neste projeto (seção \ref{sec:result:slam}). A visualização 3D do ambiente foi possível, mas ainda pode apresentar erro quando um mesmo objeto é mapeado duas vezes em tempos diferentes, pois por conta do erro da odometria a referência para gerar os dados muda.

Buscamos também, que todo o desenvolvimento aqui documentado permita um desenvolvimento aguçado nos próximos semestres, tornando a compreensão de configurações, pacotes, \textit{plugins} e sensores facilitada. As limitações e dificuldades citadas ao longo deste documento buscam motivar melhorias futuras e tornar a navegação autônoma e a visualização 3D ainda mais eficiente. Esperamos que o trabalho contribua para o desenvolvimento de uma excelência em robótica na nossa universidade. Por fim, o cronograma foi cumprido e os objetivos propostos alcalçados, conforme demonstraram os testes de navegação realizados com o robô em um ambiente interno.